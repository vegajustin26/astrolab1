\documentclass[12pt]{article} 

\usepackage[includeheadfoot, top=0.5in, bottom=0.5in, hmargin=1in]{geometry}

\usepackage{epsfig}

\usepackage{fancyhdr}
\usepackage{url}
\pagestyle{fancy}
\usepackage{setspace}
\usepackage{datetime2}
%\doublespacing
\singlespacing
%\onehalfspacing

\lhead{ASTR UN1903 -- Lab 4}
\lfoot{Vega}
\cfoot{\thepage}
\rfoot{\today}

\begin{document}\thispagestyle{empty}
 \begin{center}
{\huge Lab 4: Planet Taxonomy}\\
\end{center} 


\section*{Introduction}

What's in a name? The word planet comes from the Ancient Greek \textit{aster planetes}, meaning ``wandering star". The name made sense at the time (why?) but of course planets are not stars at all. 
\textit{Taxonomy} refers to the science of classification, and today we will explore the taxonomy of planets. Before we begin, we'll have a brief discussion on the following:

\begin{itemize}
    \item{Why is taxonomy important in science?}
    \item{Name a few definitions (in any area of science) that ascribe different classifications to similar objects. Why are the delineations drawn where they are?}
\end{itemize}

\noindent
\begin{enumerate}
    \item Briefly summarize what we discussed, and particularly note if your viewpoints had
changed over the course of our discussion.
\end{enumerate}

\section*{The IAU Needs YOU}
The International Astronomical Union (IAU) has called you up and given you the task of creating a scientific definition for the word ``planet''. Luckily, they have also provided you with a list of a bunch of the objects found that might be classified as a planet, as well as physical characteristics for each one. Unfortunately, the email they sent you cut off the names of the objects, so you're going to have to figure out the definition without knowing which is which. Compare their properties and come up with a reasonable way to classify them.

\section*{Composition}
The ``Composition'' column indicates in what form most of the mass in the body is found.  
\linebreak
\\
Find the average distance from the Sun (i.e. orbital radius) of ``rock'' objects, the average distance from the Sun of ``gas'' objects, and the average distance from the Sun of ``rock and ice'' objects.

\begin{enumerate}
    \setcounter{enumi}{1}
    \item Are there any bodies that do not seem to fit in with the others in their composition class? Which ones?
    \item If so, recalculate the average distance for that class again without this/these member(s).  Does the composition of an object strongly correlate on its distance from the Sun? 
\end{enumerate}
  

\section*{Orbital Eccentricity}
Orbital eccentricity tells us how the orbit is shaped. Astronomical bodies orbit stars in elliptical orbits, with the star at one focus. The orbital eccentricity is defined as:

\[e = \frac{c}{a}\]

\noindent Where $c$ is the distance from the center to the focus, and $a$ is the semimajor axis of the ellipse. 

\begin{enumerate}
\setcounter{enumi}{3}
    \item Sketch a quick diagram that explains this. 
    \item What shape is the orbit if e = 1? e = 0? What are the units of eccentricity?
\end{enumerate}

\noindent Identify the bodies that have the most eccentric orbits --- you can set your own ``cutoff'' value. Look at the other properties of the eccentric bodies. 

\begin{enumerate}
    \setcounter{enumi}{5}
    \item Do they have anything in common?
\end{enumerate}

\section*{Mass and Orbital Radius}
Make a scatter plot of object mass versus distance from the Sun (Orbital Radius). The x-axis will be log$_{10}(\textrm{Orbital~Radius})$ in units of AU. The y-axis will be log$_{10}(\textrm{Mass})$ in units of Earth masses ($M_E$).\\

\textbf{Helpful plotting tips: Remember to include axes labels with proper units in each plot!} Be sure that axes are bottom and left aligned (if needed in Excel: double-click an axis \textgreater \, go to Axes Options \textgreater \, Vertical/Horizontal axes crosses, and adjust the axis value). No need to include legends, since we are only plotting one series.

\begin{enumerate}
    \setcounter{enumi}{6}
    \item Why are we using logarithms for plotting? (Try plotting Mass vs. Orbital Radius without taking the log.)
    \item Do massive bodies tend to be farther or closer?
    \item What about low mass bodies? Are there any exceptions to this?
\end{enumerate}

   
\section*{Density and Orbital Radius}
Make a scatter plot of object density versus orbital radius. The x-axis will again be log$_{10}(\textrm{Orbital~Radius})$ and the y-axis will be density. 

\begin{enumerate}
    \setcounter{enumi}{9}
    \item Does there seem to be any relationship between density and distance from the Sun? 
\end{enumerate}


\section*{Moons and Mass}
Make a scatter plot of number of moons versus mass of the object. Let your x-axis be log$_{10}(\textrm{Mass})$ in units of Earth masses ($M_E$) and your y-axis be the number of moons.

\begin{enumerate}
\setcounter{enumi}{10}
    \item How do these two quantities tend to relate to each other? What does this seem to suggest to you?

\end{enumerate}

\section*{Orbital Zone}
The $\mu$ (``Mu'') column tells us how much the body has ``cleared out'' its orbital zone --- that is, collected stuff along its path around the Sun and accumulated it. More precisely, $\mu$ is the ratio of the mass of the body to
the total mass of all objects in the orbital zone.
\begin{enumerate}
\setcounter{enumi}{11}
    \item Identify the objects with the lowest $\mu$ (define your own reasonable cutoff value). Look at the other properties of these low-$\mu$ objects. Is there anything that these objects share in common?

\end{enumerate}

\section*{Classify}
Based on what have you have done (or any of the numbers on your spreadsheet), come up with your own classification scheme for these bodies.  

\begin{enumerate}
    \setcounter{enumi}{12}
    \item Do you find obvious groupings from your plots, or is there no correlation at all? You can have as many groupings as you want and as many objects in each group as you want.
    \item Are there bodies that don't seem to fit into any of your classification groups? Take your time. There is no wrong answer to this! Write down which bodies are put into each grouping, and explain your classification system in a few sentences.
    \item Now that you've come up with a classification system with different groups, which of these groups do you think should be deemed \emph{planets}? Should they all?  Should none of them?  What are your criteria for what is considered to be a planet? Write this down in a few sentences, and list the objects that you would consider to be a planet. Again, there is no wrong answer to this, just be scientific and thoughtful.
    \item When you're happy with your classification system, I will tell you the names of the objects and their IAU classifications. How do these compare to your classifications? 
\end{enumerate}

\newpage
\section*{The Case Of Pluto}
In 2006, the IAU changed the classification of Pluto from `planet' to `dwarf planet'. The public and internet memes went wild!

\begin{enumerate}
    \setcounter{enumi}{16}
\item Given what you've thought about today, what would you say to someone who asked you why Pluto is no longer a planet? Why do scientific definitions change from time to time, and why is this important for the advancement of science? It may be helpful to Google the ``IAU definition of a planet". Discuss this with a partner/group and write a reflection in your lab notebook.
\end{enumerate}

\section*{Conclusion}

\begin{enumerate}
    \setcounter{enumi}{17}
    \item What was your favorite and least favorite part of this lab?
    \item Is anything still unclear?
    \item What is your favorite planet and why?
\end{enumerate}

\end{document}