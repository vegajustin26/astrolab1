\documentclass[12pt]{article}% uses letterpaper by default

%---------- Uncomment one of them ------------------------------
\usepackage[includeheadfoot, top=0.5in, bottom=0.5in, hmargin=1in]{geometry}

% \usepackage[a5paper, landscape, twocolumn, twoside,
%    left=2cm, hmarginratio=2:1, includemp, marginparwidth=43pt, 
%    bottom=1cm, foot=.7cm, includefoot, textheight=11cm, heightrounded,
%    columnsep=1cm, dvips,  verbose]{geometry}
%---------------------------------------------------------------
\usepackage{fancyhdr}
\renewcommand{\footrulewidth}{0.4pt}% default is 0pt
\usepackage{verbatim}
\usepackage{url}
\usepackage{cancel}
\pagestyle{fancy}
\usepackage{graphicx}
\usepackage{setspace}
\usepackage{enumerate}
\singlespacing
%\doublespacing
%\onehalfspacing
%\newcommand{\exercisename}{}
\usepackage{varwidth}

\newcommand{\degrees}{\ensuremath{^\circ}}
\newcommand{\arcmin}{\ensuremath{'}}
\newcommand{\arcsec}{\ensuremath{"}}
\newcommand{\hours}{\ensuremath{^\mathrm{h}}}
\newcommand{\minutes}{\ensuremath{^\mathrm{m}}}
\newcommand{\seconds}{\ensuremath{^\mathrm{s}}}

\newcommand{\s}[0]{\phantom{i}} %sets up \s command
\newcommand{\m}[0]{\phantom{abcde}} %sets up \m command
\providecommand{\e}[1]{\ensuremath{\times 10^{#1}}} %sets up \e command
\setlength{\parindent}{0.2in} %new paragraph indent
\usepackage{indentfirst} % indent the first paragraph of a section
\usepackage{amsmath,amssymb}
\usepackage{enumitem}

\lhead{ASTR UN1903 -- Lab 10}
\lfoot{Deshmukh}
\cfoot{\thepage}
\rfoot{\today}


\begin{document}\thispagestyle{empty}

\begin{center}
    \huge Lab 10: Astrobiology
\end{center}


\section{Introduction}

Today we will be talking about astrobiology, the study of life outside of Earth. While this may sound like the stuff of science fiction, we'll try to use scientific thinking (and some simple math) to explore the possibility of extraterrestrial life. While physical experiments are crucial for scientific progress, these kinds of thought experiments often make vital tools in advancing our understanding of nature. Most of these questions do not have one correct answer, so it is important to validate and explain your thinking. The justification you provide is more important than your answer itself: the true answer remains unknown!


\section{What is Life?}

First, talk to your partner/group about these questions and write down your thoughts. We will discuss them as a class afterwards. 

\begin{enumerate}

\item How would you define life?  
How can you distinguish living things from non-living things?

\item  What is intelligent life?  
Can you distinguish intelligent life from non-intelligent life?  How? Give an example of an intelligent life-form and a non-intelligent life-form. 

%What would we need from alien civilizations for us to detect their existence or to communicate with them?

\item  What does life need to survive?  Think about the basic necessities of humans and other living things.

\item What are some of the ways in which a species can become extinct?  Name at least three of those ways.  
Are some of these fates preventable by sufficiently ``advanced'' civilizations?  Are some fates unique to them? What makes a civilization ``advanced''? Make an educated guess for the typical lifetime of an intelligent civilization based on the following table and explain your reasoning.


\begin{center}
\begin{tabular}{|l|l|}
\hline
    Age of the universe & $\sim$ 13,700,000,000 years \\ \hline
    Age of the Earth & 4,600,000,000 years \\ \hline
    Earliest fossil evidence of fossil bacteria & 3,500,000,000 years ago \\ \hline
    First multicellular fossils & 1,500,000,000 years ago \\ \hline
    Earliest invertebrates & 800,000,000 years ago \\ \hline
    Fish+amphibian domination & 590,000,000 - 248,000,000 years ago \\ \hline
    Mammals dominant & since 65,000,000 years ago \\ \hline
    Homo sapiens & originated 250,000 years ago \\ \hline
    Human civilization & $\sim$ 10,000 years old \\ \hline
    Radio communication & $\lesssim$ 100 years old \\ \hline
\end{tabular}
\end{center}

\end{enumerate}


\section{The Habitable Zone}

The habitable zone is the distance from a star at which liquid water can exist on a planet orbiting that star.  
It is called the habitable zone because we assume that life requires liquid water to survive.

\begin{enumerate}
\setcounter{enumi}{4}
\item Between what temperatures, in degrees Celsius and in Kelvin, is water liquid?  
Recall that the temperature in Kelvin is the temperature in Celsius degrees plus 273.

\item Below is an equation which relates the distance between a planet and its star, $d$, and the average surface temperature on the planet, $T$. 
Note that it also depends on the luminosity $L_{\odot}$, or energy output, of the star, since that is where planets get their heat.
$$ T = \left( \frac{L_{\odot}}{16 \pi \sigma d^{2}} \right) ^{1/4} $$
Use the equation to find the minimum and maximum distance from the Sun at which water will be liquid.  
Note: L$_{\odot}$ = $3.8 \e{33}$ ergs/second, $\sigma$ = $5.7 \e{-5}$ erg/s/cm$^{2}$/K$^{4}$.  
$T$ must be in Kelvin.  
The distance you calculate with this equation will be in cm.  

% d = 8.2 x 10^12 cm and 1.6 x 10^13 cm

\item The current distance between the Earth and the Sun is $1.5 \e{11}$ m (1 AU).  
Calculate what the average surface temperature of the Earth should be based on that distance.
% T = 277 K

\item The actual average temperature at the surface of the Earth is 15$^{\circ}$ C.  
How do your results compare to the actual value?  
If they are different, explain why that might be.

% T = 15 C = 288 K (it's hotter than it should be - global warming, greenhouse effect)

\item Repeat questions 7 and 8 for Venus, which is $1.08 \e{11}$ m (0.72 AU) from the Sun, and is actually 464$^{\circ}$ C at its surface.

\item What are the strengths and weaknesses, in your opinion, of the ``habitable zone'' concept?

\end{enumerate}


\section{Conditions for Life}
\begin{enumerate}
\setcounter{enumi}{10} 

\item Why is the habitable zone so important? Are there ways to ``cheat'' this habitable zone, i.e. can a planet be very far or close to a star and still have water?

\item Our understanding of life is somewhat (severely?) limited in that we only know of one instance of evolved life, one common ancestor on one planet, etc.  Might life exist in conditions very different from what we discussed above? What assumptions have we made that might not be true?

\item The human population on Earth doubles every 30-40 years.  Taking the population of the Earth to be 6.7 billion people at present day, and assuming a steady growth rate, estimate a rough value for the population in the year 2109.  In the year 2209?  Comment on the possibility of inhabiting Mars (or some of the Jovian moons) in solving the overpopulation problem.  The nearest star to the Sun is Proxima Centauri, 4.2 light years away.  Comment on the feasibility of colonization of other stars in solving the overpopulation problem.

\emph{Hint:  Use the following equation that governs exponential growth of a population:  A = $A_0$ $\times$ B$^{t/T}$; $A_0$ is the original value of the quantity A, B is the growth rate of quantity A, T is the time during which A increases by the growth rate B, and A gives the value of the quantity after an amount of time t.} 
\end{enumerate}

\section{The Search for Life on Other Planets}
The SETI (Search for Extraterrestrial Intelligence) Institute is constantly monitoring stars in our galaxy for signals from intelligent life on other planets.  
It is thought that the signals would be in the form of radio waves.  
Radio waves are a type of electromagnetic radiation, as is all other light.  
All electromagnetic radiation travels at the speed of light, $c=3.0 \e{8}$ m/s. 

A light-year is a unit of distance: rather unsurprisingly, it is the distance light travels in one year.  
Besides the sun, the closest star to us (Alpha Centauri) is 4.3 light years away.  
The radius of our galaxy is 51,000 light-years.  

\begin{enumerate}
\setcounter{enumi}{13}
\item How long would it take for a signal from Alpha Centauri to reach us?  
How long would it take for a signal from 51,000 light-years away to reach us?  (Hint: No calculation needed.)%  To convince yourself: 1 light-year = 9.461 $\times$ 10$^{15}$ m.  Distance = velocity $\times$ time.)
\item  If we learn tomorrow that SETI has detected signals from a star-planet system 51,000 light-years away, do you think the civilization responsible for the signal is currently more or less advanced than ours?  Why? %  (Hint: Human civilization has existed for about 10,000 years and has had radio technology for less than 100 years.)

\item Speculate on the feasibility of having a ``conversation'' with an alien civilization. 

\end{enumerate}


\section{Drake's Equation}

Astronomer Frank Drake created the following equation to determine the number of intelligent civilizations that are willing and able to communicate:
$$ N = R_{*} \times f_{p} \times n_{e} \times f_{L} \times f_{I} \times f_{C} \times L. $$
%\begin{flushleft}
$R_{*}$ is the rate of star formation in our galaxy.
\\
$f_{p}$ is the fraction of stars that have planets.
\\
$n_{e}$ is the average number of habitable planets for every star that has planets.
\\
$f_{L}$ is the fraction of habitable planets that actually develop life.
\\
$f_{I}$ is the fraction of the above with intelligent life.
\\
$f_{C}$ is the fraction of the above that develop technology that releases detectable signs of their existence into space.
\\
$L$ is the expected lifetime of such a civilization.

\begin{enumerate}
\setcounter{enumi}{16}
\item  What does \emph{N} represent?  
Show that it has proper units by writing the equation out explicitly with the units of all the variables. 
%ne is dimensionless!


\item  $R_{*}$ is about 10 stars per year.  
Make educated guesses for the values of the other parameters.  \emph{Explain your reasoning in each case.}

\item  What do you get for $N$?  How does it compare to the observed value of $N$?

\item 
Why is the Drake equation useful? What might it be missing??
\end{enumerate}


\section{Conclusions}

\begin{enumerate}
\setcounter{enumi}{20}
\item Extremophiles are organisms on Earth that live in extreme conditions such as the depths of the ocean where no light can penetrate or extremely cold climates.  
What does their existence on earth suggest about life on other planets?  

\item Most stars in the galaxy are less massive than the Sun, meaning they are smaller and do not output as much energy. 
Based on this fact, would you expect planets that harbor life to be closer to or further from their host star than the Earth is to the Sun. 

\item What did you like or dislike about this lab? What did you find interesting?

\item Any feedback/suggestions?

\end{enumerate}






\end{document}