\documentclass[12pt]{article}% uses letterpaper by default

%---------- Uncomment one of them ------------------------------
\usepackage[includeheadfoot, top=0.5in, bottom=1in, hmargin=1in]{geometry}

% \usepackage[a5paper, landscape, twocolumn, twoside,
%    left=2cm, hmarginratio=2:1, includemp, marginparwidth=43pt, 
%    bottom=1cm, foot=.7cm, includefoot, textheight=11cm, heightrounded,
%    columnsep=1cm, dvips,  verbose]{geometry}
%---------------------------------------------------------------
\usepackage{fancyhdr}
\usepackage{verbatim}
\usepackage{url}
\pagestyle{fancy}
\usepackage{graphicx}
\usepackage{setspace}
\usepackage{enumerate}
\usepackage{enumitem}
%\doublespacing
\singlespacing
%\onehalfspacing
%\newcommand{\exercisename}{}

\usepackage{datetime2}  % customize \today, defaults to yyyy-mm-dd
\cfoot{\thepage}
\rfoot{\today}
\renewcommand{\rightmark}{}
\renewcommand{\headrulewidth}{0pt}
\renewcommand{\footrulewidth}{0.4pt}

\newcommand{\degrees}{\ensuremath{^\circ}}
\newcommand{\arcmin}{\ensuremath{'}}
\newcommand{\arcsec}{\ensuremath{"}}
\newcommand{\hours}{\ensuremath{^\mathrm{h}}}
\newcommand{\minutes}{\ensuremath{^\mathrm{m}}}
\newcommand{\seconds}{\ensuremath{^\mathrm{s}}}

\begin{document}
\begin{center}
\huge{Meteorites and the Early Solar System}
\end{center}



\vspace{0.3cm}

\begin{flushleft}

There's a lot more to the Solar System than just planets! In the Hall of
Meteorites at the Museum of Natural History, we can learn about the other solid
bodies that orbit the Sun. Although smaller, they have played a role in the
evolution of life on Earth, and they carry records that tell the history of the
Solar System going all the way back to the formation of the planets and even
further.  

\vspace{0.3cm}

\textbf{\emph{After} taking time to explore all of the exhibits in the 
Hall of Meteorites, 
\emph{then} answer the questions below in your lab book. Some answers require 
you to combine information from more than one exhibit.} 

\vspace{0.3cm}

\begin{center}
\textbf{I. The Ahnighito Meteorite}
\end{center}

\begin{enumerate}
\item What makes this meteorite so special for it to be the centerpiece of the 
collection here at AMNH?

\vspace{0.3cm}

\item Draw a diagram of the relative size of the Ahnighito meteorite by making a rough
sketch of its profile, along with a person standing next to it. 

\vspace{0.3cm}

\item What is the meteorite mostly made of?  

\item How heavy is the Ahnighito meteorite?
If you assume that the average person weighs 150 lbs, and a ton is 2000 lbs, then how many people would you need to be as heavy as this relatively compact 
meteorite?

\vspace{0.3cm}

\item What is the difference between a meteorite and a meteor? 

\vspace{0.3cm}

\item Give an example of a good place on the Earth to search for meteorites.

\vspace{0.3cm}

\begin{center}
\textbf{II. Impacts}
\end{center}

\item Why is the Moon's face covered in craters, while we only see a few on Earth? 
Explain.

\vspace{0.3cm}

\item How does the impact of a large body lead to a mass extinction, as in the
case of the dinosaurs?  

\vspace{0.3cm}

\item When and where do scientists think Chicxulub (the dinosaur-killing impact) 
occurred?  What is the evidence for it being responsible for this mass
extinction?

\vspace{0.3cm}

\item Describe one plausible way to divert the path of an asteroid on a collision
course with Earth.

\vspace{0.3cm}

\item Briefly explain the leading theory of how the Moon formed. 

\vspace{0.3cm}

% \newpage

\begin{center}
\textbf{III. The Early Solar System}
\end{center}

\item What is the solar nebula? 

\vspace{0.3cm}

\item All in all, what do \emph{you} consider to be the main reasons scientists
would want to study meteors/meteorites?  Please cite two reasons.

\vspace{0.3cm}

\begin{center}
\textbf{IV. Passport to the Universe Space Show (Post space-show questions)}
\end{center}

\item What kinds of astronomical objects/structures did we take a tour of? List some of them in order of smallest to largest scale.

\vspace{0.3cm}

\item What did you learn about from the space show? 

\vspace{0.3cm}

\begin{center}
\textbf{V. Exhibit Study}
\end{center}

\item Find an exhibit that we have not looked at yet and reflect on the following:
\begin{enumerate}[label = \alph*.]
\item What is the exhibit?
\item What information did the curator try to convey and how did they go about doing this?
\item What did this exhibit do correctly? (e.g., how did they effectively communicate the science to a broad audience, such as those that attend a museum).
\item Was there anything that you felt could have been more effective? How would have you redesigned this exhibit if you were in charge of creating it?
\end{enumerate}
\vspace{0.3cm}
\begin{center}
\textbf{VI. Reflection}
\end{center}


\item Write a brief (2-4 sentences) reflection on what you took away from this lab. If you need some inspiration, what was your favorite part of visiting The Rose Center for Earth and Space and why? Alternatively/in addition, describe something new you learned.

\end{enumerate}
\end{flushleft}
\end{document}