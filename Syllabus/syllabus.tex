\documentclass[11pt]{article}
\usepackage[includeheadfoot, top=1.0in, bottom=1.0in, hmargin=1.0in]{geometry}
\usepackage[utf8]{inputenc}
\usepackage{fancyhdr}
\usepackage{url}
\usepackage{tgtermes}
\pagestyle{fancy}
\usepackage{setspace}
\usepackage{tabularx}
\usepackage{hyperref}
\usepackage{xcolor}

\lhead{Astronomy Lab I}
\rhead{Spring 2025}
\cfoot{\thepage}

\begin{document}\thispagestyle{empty}
\title{\vspace{-2.5cm}\textbf{ASTR1903: Astronomy Lab I}}
\date{}
\maketitle
\vspace{-60pt}
\begin{center}
    \Large{Syllabus}
\end{center}
% \begin{center}
% \textbf{\LARGE{}}\\ \medskip \Large{Syllabus}\\ 
% \end{center}
\vspace{12pt}
\noindent
\textbf{Instructor:} Justin Vega, \textit{he/him}
(j.vega@columbia.edu)\\
\textbf{Office:} Pupin 1410 (office hours by appointment)\\ 
\textbf{Time:} {Tuesdays, 7:00-10:00 PM} \\
\textbf{Location:} {Astronomy Library, Pupin 1402 \\}
\vspace{-12pt}
\section*{Course Overview}
Welcome to Astronomy Lab I! This lab corresponds in part to topics discussed in the \textit{Life in the Universe}; \textit{Earth, Moon, Planets}; and \textit{Another Earth} courses. There will be 10 labs throughout the semester, one class for make-up labs, and two classes for formulating and presenting your final projects on an astronomy topic of your choice. \\
\\
The first portion of each class will be lecture/discussion-based, introducing you to key concepts that you'll need to be familiar with to complete the assigned lab. You will then work with a lab partner/small group on the lab. You are expected to complete the majority, if not the entirety of your lab during class time – work outside the class is not required, but may be necessary depending on your lab progress.
\\
\\
\noindent The goals of this lab course include the following:
\begin{itemize}
\item Developing a better sense of the universe and its extraordinary phenomena, understanding how we begin to study the cosmos with quantitative tools and measurements

\item Demystifying the scientific method, developing critical thinking skills, and learning to apply scientific reasoning in your evaluation of information and arguments

\item Experiencing the scientific process of framing and asking a well-defined question, gathering data and testing that question quantitatively, and communicating your work -- through writing and speech -- to your peers
\end{itemize}

\section*{Lab Materials}
 
The only specific materials you need are the following:
 
\begin{itemize}
\item \textbf{Lab notebook:} This can be a physical notebook, or an electronic document instead; if you choose the latter, make sure to have your device with you and ready to use.

\item \textbf{Tablet/Laptop computer:} Laptops will be a necessity for many of the labs. Please ensure your device is charged and ready to use before class. A limited number of laptops will be available for students who don't have their own - if you need one, please let me know before class if possible. \\
\end{itemize}
\vspace{-12pt}
\noindent Labs will be posted to Courseworks at least two hours before class starts. I will only provide paper copies of labs if requested.
%%%%%%%%%%%%%%%%%%%%%%% GRADING %%%%%%%%%%%%%%%%%%%%%%%
\section*{Grading}

\subsection*{Lab Write-ups}
Each lab will clearly denote what you should record in your write-ups for each lab. Lab responses can be recorded in either a bound physical notebook or in an electronic document. You may submit your work either as a \textbf{PDF} to Courseworks (\textit{strongly preferred}), or hand in your lab notebook to me at the end of class, to be returned at beginning of the following lab.  All submissions will be due by midnight on the day following the lab (Wednesday at 11:59PM). \\
\\
\noindent While you will be working with a lab partner, each of you should keep your own records, and submit your own write-up. The entire goal of the write-ups is to explain to me \textit{what} you did during the lab, \textit{how} you did it, and \textit{why} you did it --- I am much more concerned with your reasoning behind your arguments than I am about the format. \\

\noindent Here are some general guidelines for lab write-ups:
\begin{itemize}
\item[--] Each lab's write-up should begin on a new page and have your name, your partner's name, lab title, and the date at the top.
\item[--] Use complete sentences to communicate your assumptions, observations, and conclusions. Show all work for your calculations.

\item[--] Indicate quantities with appropriate units (when necessary); label plots with axes titles, and diagrams with features.

\end{itemize}

\subsection*{Participation}
\noindent As science is a collaborative discipline, you will be required to work with a partner for each lab. Your participation grade will be based on your contributions to your lab group, your class attendance, and your participation in class discussion. You are always encouraged to ask questions, regardless of content. You will also have the opportunity to record lingering questions in your lab write-ups.

\smallskip

\subsection*{Final Presentations}
\noindent For the final session, each student will give a 10-minute presentation on a topic or their choice followed by a 5-minute discussion with the class. A list of topics related to astronomy and science in society will be provided, but you are also welcome to submit your own suggested topics, pending my approval.

\subsection*{Grade Breakdown}

\noindent \textbf{65\%} Lab submissions*

\noindent \textbf{20\%} Final Presentations

\noindent \textbf{15\%} Participation \\
\\
\noindent *Your lowest lab grade will be dropped when determining your final grade.

%%%%%%%%%%%%%%%%%%%%%%% SCHEDULE %%%%%%%%%%%%%%%%%%%%%%%
\section*{Tentative Schedule \small{(last updated \today)}} 

\noindent
\begin{tabular}{cc}
    1/28 & Syllabus, Lab 1: Orders of Magnitude\\ 
    2/4 & Lab 2:  The Height of Pupin\\
    2/11 & Lab 3: Pseudoscience \\
    2/18 & Lab 4: Planet Taxonomy \\
    2/25 & Lab 5: Observing Lab\\
    3/4 & Lab 6: Exoplanets \\
    3/11 &  Lab 7: The Sun\\
    3/18 & \textit{Spring Recess, no lab}\\
    3/25 & Final Project Prep \\
    4/1 & Lab 8: Astronomical Coordinates  \\
    4/8 & Lab 9: Multiwavelength Universe \\
    4/15 & Lab 10: Astrobiology \\
    4/22 & Makeup Lab \\
    4/29 & Presentations  \\
    
\end{tabular}


%%%%%%%%%%%%%%%%%%%%%%% POLICIES %%%%%%%%%%%%%%%%%%%%%%%
\section*{Policies}
 
\subsection*{Attendance}
 
Attendance to every lab session is required– you will be expected to attend every class session for the duration of class time. If your group finishes the lab early, you may turn it in and leave early. The class session will not be extended if a student is late. By department policy, more than two unexcused (non-medical related) absences will result in automatic failure of the course. Please notify me if extenuating circumstances arise (family emergencies, serious illness, quarantine requirement, or religious holidays) and we will arrange a make-up lab.

% {
% \color{red}
% \noindent There will be one week for make-up labs. Attendance is not required for this class, but note that it is in the middle of the semester, as our course schedule does not allow for additional make-ups, so attendance is strongly encouraged. Please notify me as soon as possible if extenuating circumstances arise that will affect your attendance so that we can determine the best course of action. \\
%  }
\subsection*{Accommodations}
Please speak with me if this course can be better adapted to your needs without sacrificing the integrity of instruction. If you have an identified disability, we encourage you to register with the Office of Disability Services early in the semester to ensure access to any necessary resources; registration is confidential.

\subsection*{Academic Honesty}
Please write your own lab reports, without copying from others or using AI tools (e.g. ChatGPT). I'm more than happy to help you along with the process, and you get so much more out of it by doing it yourself.
 
\subsection*{Mandatory reporting}
Instructors are required to report allegations of gender based misconduct, discrimination, or harassment to Columbia's administration. While I am willing to listen and seek out resources (including confidential counselors) on your behalf, I cannot myself provide confidentiality.

\end{document}
