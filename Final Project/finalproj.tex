\documentclass[12pt]{article}
\usepackage[includeheadfoot, top=0.5in, bottom=0.5in, hmargin=1in]{geometry}
\usepackage[utf8]{inputenc}
\usepackage{fancyhdr}
\usepackage{url}
\pagestyle{fancy}
\usepackage{setspace}
\usepackage{tabularx}
\usepackage{graphicx}
\usepackage{caption}
\usepackage{subcaption}
\usepackage{hyperref}
\usepackage{multicol}
\usepackage{amsmath}
\usepackage{enumitem}
\usepackage{outlines}
\usepackage{datetime2}

\usepackage{hyperref}
\hypersetup{
    colorlinks=true,
    linkcolor=blue,
    filecolor=magenta,      
    urlcolor=blue,
}


\lhead{Final Project}
\lfoot{Vega}
\cfoot{\thepage}
\rfoot{\today}

\begin{document}\thispagestyle{empty}

\begin{center}
\huge{Final Project}\\ \medskip 
\end{center}

\setstretch{1.15}
\section{Overview}
In this course, we have only begun to unravel the mysteries of the universe (in the form of neat and concise little labs, for the most part). However, astronomy as a discipline is vast, prolific, confusing, oftentimes messy — and it can lead to the most mind-boggling discoveries of our time. There are many opportunities for deep, focused exploration within the field of astronomy, but there are also many pathways to be explored between astronomy and other disciplines. These ideas of \textit{deep exploration} and \textit{interdisciplinary connections} will be the guiding theme for our final projects.

For our last class on April 29th, you will present a project on an astronomy topic of your choice that focuses on the theme of exploration and/or connections to other fields. I've provided a list of potential projects and topics in a separate document, but I highly encourage you to be creative and draw from your own curiosities and interests (if it's not fun and interesting to you, it won't be fun for me grading it!).

One approach could be to show how you can connect astronomy to your major. For example, a political science major may find doing a final project on space policy interesting, while someone studying economics could investigate how large telescope missions get funded. But again, this is not a strict requirement— I'm just giving you some starting points to work with. If you're particularly stuck, I am happy to work with you to help think of a direction to take your project.

\section{Guidelines}
\subsection*{Deliverables}
\begin{outline}
\setstretch{1}
    \1 Submit a short write-up of your project plan on Courseworks by \textbf{April 1st} (no joke!), describing your proposed topic and medium. I'll approve the project and give feedback as needed so that you can get started on it.
        \2 You may work in small groups of 2-3, but each person will be graded for their (roughly equal) contribution to the project.
        \2  If working in a group, you can upload the same document, but outline what each person will specifically contribute to the project.
    \1 Prepare a short presentation about your project for class on \textbf{Tuesday, April 29th}. 
        \2 If you'll be using slides, you should submit your presentation slides by midnight on \textbf{Monday, April 28th} (i.e., the night before the presentations). Where appropriate, you should also include references in your slides (e.g., research papers, popular science articles, websites, books, etc.); no special formatting or citation style is needed.
    \1 Prepare a reflection on your project to submit on Courseworks by the last day of classes, \textbf{Monday, May 5th}.
\end{outline}

\subsection*{For the Presentation}

\begin{itemize}[noitemsep]
    \item Presentations should be \textbf{$\sim$10-15 min if you are a group of 2-3, or $\sim$7 min if you are an individual}
    \item Each presentation will be followed by a \textbf{3 minute question period}.
    \item All presentations should include a description of the science underlying your chosen topic, as well as the limitations of our collective knowledge on this topic (see rubric under ``Grading" section).
    \item For research project presentations, you should choose a topic that we have not covered \textbf{in detail} in class. You should be learning something new! For your presentation, you may use any combination of slides and/or whiteboard.
    \item For non-research projects (i.e., projects that creatively interpret a topic we've covered in class), you should also include in your presentation:
    \begin{itemize}
        \item A description and presentation of the project itself
        \item Why you chose this specific medium and how you are using it to convey key information to your audience.
    \end{itemize}
    \item Come ready to ask questions during and after each presentation; questions will count for a participation grade.  Any type of question is welcome (e.g., asking the presenter to clarify a statement, asking the presenter for more background information, asking the presenter hypothetical questions based on relevant scenarios) -- remember, there's no such thing as a bad question!
\end{itemize}

\subsection*{For the Reflections}
\begin{itemize}[noitemsep]
    \item Reflections should be \textbf{at least 1 page in length}
    \item Each reflection should include the following:
    \begin{itemize}
        \item A description of the underlying science concept and how this concept is connected to previously covered course material. (2 paragraphs)
        \item A description of the project itself. (1 paragraph)
        \item A description of the key information you want to convey to your audience (1 paragraph)
        \item An explanation of why you chose this particular medium and this particular representation of the concept, as well as a justification for why you think this mode of communication is particularly effective (1-2 paragraphs)
        \item A summary of what you learned from this project and what else you would like to learn about this topic (1 paragraph)

    \end{itemize}
\end{itemize}

\subsection*{Grading}
The final project is worth 20\% of your final course grade. You will be graded on your presentation and your reflection, with the following rubric/point breakdowns.

\subsubsection*{Presentation (100 pts)}

\textbf{Content - 70 pts}
\begin{itemize}[leftmargin=0.3in]
%\itemsep0em
\item Presenter introduces and describes topic at level appropriate to this class [\underline{\hspace{5mm}}]
\item Presenter explains extent of and limitations on our knowledge of the topic, including data/observations underlying knowledge [\underline{\hspace{5mm}}]
\item Presenter provides context by drawing connections to, e.g., different areas of astronomy, concepts from lab or
lecture, other areas of science, areas outside of science, etc. [\underline{\hspace{5mm}}]
\item Presenter chooses and cites appropriate references (i.e., goes beyond Wikipedia and popular press releases).  Presenter submits reference list. [\underline{\hspace{5mm}}]
\end{itemize} 


\noindent \textbf{Delivery - 30 pts}
\begin{itemize}[leftmargin=0.3in]
\item Presentation has a logical flow that audience can follow [\underline{\hspace{5mm}}]
\item Presenter can address reasonable audience questions [\underline{\hspace{5mm}}]
\item Presentation aids (slides or board-work) or final creative project are understood by audience [\underline{\hspace{5mm}}]
\item Presenter stays within allotted time [\underline{\hspace{5mm}}]
\item Presenter speaks clearly, and keeps the audience engaged (with, e.g., questions, activities, etc.) [\underline{\hspace{5mm}}]
\end{itemize}
{\small [\underline{\hspace{5mm}}] = easily and concisely (4), sufficiently (3),
is somewhat able to (2), barely/did not (1)
}
\subsubsection*{Reflection (40 pts)}
\noindent
The reflections will be graded with the following point breakdown: 20 points for depth/ thoroughness of explanations, 10 points for clarity of writing, and 10 points for correctness.


\end{document}

