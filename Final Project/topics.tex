\documentclass[12pt]{article}
\usepackage[includeheadfoot, top=0.5in, bottom=0.5in, hmargin=1in]{geometry}
\usepackage[utf8]{inputenc}
\usepackage{fancyhdr}
\usepackage{url}
\pagestyle{fancy}
\usepackage{setspace}
\usepackage{tabularx}
\usepackage{graphicx}
\usepackage{caption}
\usepackage{subcaption}
\usepackage{hyperref}
\usepackage{multicol}
\usepackage{amsmath}
\usepackage{enumitem}
\usepackage{outlines}
\usepackage{datetime2}

\usepackage{hyperref}
\hypersetup{
    colorlinks=true,
    linkcolor=blue,
    filecolor=magenta,      
    urlcolor=blue,
}


\lhead{Final Project}
\lfoot{Vega}
\cfoot{\thepage}
\rfoot{\today}

\begin{document}\thispagestyle{empty}

\begin{center}
\huge{Final Project}\\ \medskip 
\end{center}

\setstretch{1.15}

\section{Potential topics}
 \noindent
 Below is a list of potential project media as well as a non-comprehensive list of suggested topics.  You can choose something not listed, so long as it's relevant to the topics we've covered in lab (e.g., exoplanets, stars, galaxies, cosmology, etc.). If you are doing a research project, you should choose a topic that you haven't covered in depth in class or in this lab.

\medskip \noindent
More focused/specific topics often yield more compelling presentations (and are often better suited for short presentations). A sufficiently specific topic would be something like ``The Great Red Spot and other storms on Jupiter,'' while something like ``Gas giant atmospheres'' would require more specificity.

\bigskip \noindent
Example Project Ideas:
\begin{itemize}[noitemsep]
    \item A research project (with an associated PowerPoint and/or whiteboard presentation) on a topic we haven't covered in class (e.g., a new astronomy concept, an astronomer/scientist we haven’t discussed in class, an instrument/observatory/technique we haven’t discussed in class, etc.)
    \item A description of a museum exhibit that you'd design to teach the public about a specific concept
    \item A short performance or dialogue
    \item A visual, audio, or mixed-media art piece (sculptures, music, \textit{etc.})
    \item A creative writing piece (e.g., poetry or a short story)
    \item Culinary arts (e.g., baked goods representing some astronomy concept, a ``cookbook" on how to create stars/planets/galaxies)

\end{itemize}

\medskip \noindent
Topic Ideas:
\begin{itemize}[noitemsep]
    \item Galaxies (including our own)
        \begin{itemize}[noitemsep]
            \item Galactic dynamics (e.g., birth, growth, rotation of galaxies)
            \item Supermassive black holes
            \item Different theories of dark matter (or different dark matter candidates)
            \item The intergalactic medium (IGM)
            \item Dark matter halos and the dark matter content of different galaxies
            \item Dwarf galaxy satellites of the Milky Way
            \item Ultra-faint dwarf galaxies
            %\item Stellar life cycle---from birth to supernova!
            \item Dark energy
            \item Galaxy clusters
        \end{itemize}

    \item Stars (including our Sun)
        \begin{itemize}[noitemsep]
            \item Interior structure and chemistry of stars
            \item Asteroseismology or helioseismology 
            \item Stellar atmospheres or magnetospheres
            \item Stellar or solar winds
            \item The process of star formation (or the properties of star-forming regions in galaxies)
            %\item Stellar life cycle---from birth to supernova!
            \item Binary star systems
            \item Clusters of stars (globular clusters or open clusters)
            \item Specific types of star (e.g., T Tauri, RR Lyrae, Population III (the first stars))
        \end{itemize}

    \item (Exo)Planets
        \begin{itemize}[noitemsep]
            \item Solar system formation and history
            \item Planet X
            \item Proto-planetary disks
            \item Planet and planetesimal formation
            \item Brown dwarfs
            \item Exoplanet detection methods not discussed in class (e.g., microlensing, astrometry)
            \item Exoplanet atmospheres
        \end{itemize}
    
    \item Astrobiology
    \begin{itemize}[noitemsep]
        \item The Search for Extraterrestrial Life (SETI)
        \item The Drake equation
        \item Dyson spheres (or other hypothetical megastructures)
        \item Technosignatures vs. Biosignatures
        \item Communication and signal detection; candidate SETI signals
        \item Breakthrough Listen or Breakthrough Starshot
    \end{itemize}
    
    \item Telescopes and spacecrafts
        \begin{itemize}[noitemsep]
            \item Specific missions/projects (e.g., Hubble Space Telescope, James Webb Space Telescope, Kepler, TESS, Nancy Grace Roman Space Telescope, Vera C. Rubin Observatory, Thirty Meter Telescope).
            \item Astronomy at specific wavelengths (e.g., Radio astronomy and very-long-baseline interferometry (VLBI), sub-millimeter astronomy, X-ray astronomy, gamma-ray astronomy)
            \item NASA budget, missions, proposals (i.e., how funding decisions are made)
            \item Space policy (i.e., laws governing space)
        \end{itemize}

    \item Controversial Astronomy
        \begin{itemize}[noitemsep]
            \item Planet X
            \item Extraterrestrial life 
            \item Phosphine on Venus
        \end{itemize}
    \item Science and Society
        \begin{itemize}[noitemsep]
            \item A biographical presentation on a famous astronomer. If you do this, choose 1-2 scientific contributions to emphasize. Some suggestions for scientists:
            \begin{itemize}[noitemsep]
                \item Annie Jump Cannon (spectra of stars)
                \item Cecilia Payne-Gaposchkin (the composition of stars)
                \item Vera Rubin (dark matter)
                \item Jocelyn Bell Burnell (radio pulsars)
                \item Nancy Grace Roman (stellar classification and motion)
                \item Jill Tarter (SETI)
                \item Sara Seager (exoplanets)
                \item Caroline Herschel (comets)
                \item Annie Maunder (sunspots, solar corona, eclipses)
                \item Margaret Kivelson (solar wind, Europa’s ocean)
            \end{itemize}
    \end{itemize}
    \item Miscellaneous
        \begin{itemize}[noitemsep]
            \item The Big Bang and the early Universe (e.g., inflation, nucleosynthesis, the epoch of recombination, the epoch of reionization)
            \item The cosmic microwave background (CMB)
            \item Gravitational waves and LIGO
            \item Compact objects (Black holes, neutron stars, pulsars, magnetars, white dwarfs)
            \item High-energy explosions (Fast Radio Bursts or Gamma-Ray Bursts)
            
                
            \end{itemize}
            
            \item A recent or historically significant astronomy paper (I recommend searching through \url{https://ui.adsabs.harvard.edu/} or \url{https://arxiv.org/archive/astro-ph}, or asking me for help finding a paper).
        \end{itemize}
        
\end{itemize}


\end{document}

